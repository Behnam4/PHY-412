\documentclass[fleqn]{article}
\oddsidemargin 0.0in
\textwidth 6.0in
\thispagestyle{empty}
\usepackage{import}
\usepackage{amsmath}
\usepackage{graphicx}
\usepackage{flexisym}
\usepackage{calligra}
\usepackage{amssymb}
\usepackage{bigints} 
\usepackage[english]{babel}
\usepackage[utf8x]{inputenc}
\usepackage{float}
\usepackage[colorinlistoftodos]{todonotes}


\DeclareMathAlphabet{\mathcalligra}{T1}{calligra}{m}{n}
\DeclareFontShape{T1}{calligra}{m}{n}{<->s*[2.2]callig15}{}
\newcommand{\scriptr}{\mathcalligra{r}\,}
\newcommand{\boldscriptr}{\pmb{\mathcalligra{r}}\,}

\definecolor{hwColor}{HTML}{442020}

\begin{document}

  \begin{titlepage}

    \newcommand{\HRule}{\rule{\linewidth}{0.5mm}}

    \center

    \begin{center}
      \includegraphics[height=11cm, width=11cm]{asu.png}
    \end{center}

    \vline

    \textsc{\LARGE Classical Parts/Field/Matter III}\\[1.5cm]

    \HRule \\[0.5cm]
    { \huge \bfseries Homework Three}\\[0.4cm] 
    \HRule \\[1.0cm]

    \textbf{Behnam Amiri}

    \bigbreak

    \textbf{Prof: Samuel Teitelbaum}

    \bigbreak

    \textbf{{\large \today}\\[2cm]}

    \vfill

  \end{titlepage}

  \begin{enumerate}
    \item \textbf{9.8 (30 points)} Equation $9.36$ describes the most general \textbf{linearly} polarized wave on a string. 
    Linear (or “plane”) polarization (so called because the displacement is parallel to a fixed vector $\hat{n}$) results
    from the combination of horizontally and vertically polarized waves of the \emph{same phase} (Eq. $9.39$). If the two 
    components are of equal amplitude, but out of phase by $90^{\circ}$ (say, $\delta_v=0$, $\delta_h=90^{\circ}$), the result is a circularly
    polarized wave. In that case:
    \begin{enumerate}
      \item At a fixed point $z$, show that the string moves in a circle about the $z$ axis. Does it go \emph{clockwise} or 
      \emph{counterclockwise}, as you look down the axis toward the origin? How would you construct a wave circling the 
      other way? (In optics, the clockwise case is called \textbf{right circular polarization}, and the counterclockwise, \textbf{left
      circular polarization}.)
      \\
      \emph{An elegant notation for circular polarization (or elliptical, if the amplitudes are unequal) is to use a
      complex $\hat{n}$, but I shall not do so in this book.}

        % \textcolor{hwColor}{
        %   \\
        % }

      \item Sketch the string at time $t = 0$.
      
        % \textcolor{hwColor}{
        %   \\
        % }


      \item How would you shake the string in order to produce a circularly polarized wave?
      
        % \textcolor{hwColor}{
        %   \\
        % }

    \end{enumerate}


    \item \textbf{9.13 (40 points)} Find all elements of the Maxwell stress tensor for a monochromatic plane wave traveling in the 
    $z$ direction and linearly polarized in the $x$ direction (Eq. $9.48$). Does your answer make sense? (Remember that 
    $-\overleftrightarrow{T}$ represents the momentum flux density.) How is the momentum flux density related to the energy
    density, in this case?

      % \textcolor{hwColor}{
      %   \\
      % }


    \item \textbf{9.14 (30 points)} Calculate the exact reflection and transmission coefficients, without assuming $\mu_1=\mu_2=\mu_0$. 
    Confirm that $R+T=1$.

      % \textcolor{hwColor}{
      %   \\
      % }


    \item \textbf{9.7 (10 points)} Suppose string $2$ is embedded in a viscous medium (such as molasses), which imposes a drag force 
    that is proportional to its (transverse) speed:
    $$
      \Delta F_{drag}=-\gamma \dfrac{\partial f}{\partial t} ~ \Delta z
    $$
    \begin{enumerate}
      \item Derive the modified wave equation describing the motion of the string.

        % \textcolor{hwColor}{
        %   \\
        % }

      \item Solve this equation, assuming the string vibrates at the incident frequency $\omega$. That is, look for solutions of the
      form $\tilde{f}(z,t)=e^{i \omega t} ~ \tilde{F}(z)$.

        % \textcolor{hwColor}{
        %   \\
        % }

      \item Show that the waves are \textbf{attenuated} (that is, their amplitude decreases with increasing $z$). Find the characteristic penetration distance, at which the amplitude
      is $1/e$ of its original value, in terms of $\gamma$ , $T$, $\mu$, and $\omega$.

        % \textcolor{hwColor}{
        %   \\
        % }

      \item If a wave of amplitude $A_I$ , phase $\delta_I=0$, and frequency $\omega$ is incident from the left (string 1), find the reflected wave’s
      amplitude and phase.

        % \textcolor{hwColor}{
        %   \\
        % }

    \end{enumerate}


  \end{enumerate}

\end{document}
