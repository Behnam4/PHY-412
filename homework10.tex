\documentclass[fleqn]{article}
\oddsidemargin 0.0in
\textwidth 6.0in
\thispagestyle{empty}
\usepackage{import}
\usepackage{amsmath}
\usepackage{graphicx}
\usepackage{flexisym}
\usepackage{calligra}
\usepackage{amssymb}
\usepackage{bigints} 
\usepackage[english]{babel}
\usepackage[utf8x]{inputenc}
\usepackage{float}
\usepackage[colorinlistoftodos]{todonotes}


\DeclareMathAlphabet{\mathcalligra}{T1}{calligra}{m}{n}
\DeclareFontShape{T1}{calligra}{m}{n}{<->s*[2.2]callig15}{}
\newcommand{\scriptr}{\mathcalligra{r}\,}
\newcommand{\boldscriptr}{\pmb{\mathcalligra{r}}\,}

\definecolor{hwColor}{HTML}{442020}

\begin{document}

  \begin{titlepage}

    \newcommand{\HRule}{\rule{\linewidth}{0.5mm}}

    \center

    \begin{center}
      \includegraphics[height=11cm, width=11cm]{asu.png}
    \end{center}

    \vline

    \textsc{\LARGE Classical Parts/Field/Matter III}\\[1.5cm]

    \HRule \\[0.5cm]
    { \huge \bfseries Homework 10}\\[0.4cm] 
    \HRule \\[1.0cm]

    \textbf{Behnam Amiri}

    \bigbreak

    \textbf{Prof: Samuel Teitelbaum}

    \bigbreak

    \textbf{{\large \today}\\[2cm]}

    \vfill

  \end{titlepage}

  \begin{enumerate}
    \item \textbf{11.15 (40 points)} Find the angle $\theta_{max}$ at which the maximum radiation is emitted, in
    Ex. 11.3 (Fig. $11.13$). Show that for ultrarelativistic speeds (v close to c), $\theta_{max} \approxeq \sqrt{(1-\beta)/2}$. 
    What is the intensity of the radiation in this maximal direction (in the ultrarelativistic case), in proportion 
    to the same quantity for a particle instantaneously at rest? Give your answer in terms of $\gamma$.
    \emph{Note that this is for on-axis radiation (i.e. Bremstrahlung) which is a bit different from the example we 
    did in class.  Start with Eq. 11.74 in Griffiths, and proceed as we did in class to calculate the emitted intensity.}

      \textcolor{hwColor}{
        \\
        Starting with equation $11.74$ as it was suggested we have the following:
        \\
        \\
        $
          \dfrac{dP}{d\Omega}=\dfrac{\mu_0 q^2 a^2}{16 \pi^2 c} \dfrac{sin^2 \theta}{(1-\beta cos \theta)^5}
          \\
          \\
        $
        We learned from Calculus how to find the maxima and minima for an equation by differentiation.
        \\
        \\
        $  
          \dfrac{d}{d\theta} \left[\dfrac{sin^2 \theta}{(1-\beta cos \theta)^5}\right]=0
          \\
          \\
          \\
          \dfrac{
            2 cos \theta sin \theta (1-\beta cos \theta)^5- sin^2 \theta \left[5 (0+\beta sin \theta) (1-\beta cos \theta)^4 \right]
          }{
            (1-\beta cos \theta)^{10}
          }=0
          \\
          \\
          \\
          \dfrac{
            2 cos \theta sin \theta (1-\beta cos \theta)^5- 5 \beta sin^3 \theta (1-\beta cos \theta)^4
          }{
            (1-\beta cos \theta)^{10}
          }=0
          \\
          \\
          \\
          \dfrac{2 cos \theta sin \theta}{(1-\beta cos \theta)^5}-\dfrac{5 \beta sin^3 \theta}{(1-\beta cos \theta)^6}=0
          \\
          \\
          \\
          \dfrac{2 cos \theta sin \theta}{(1-\beta cos \theta)^5}=\dfrac{5 \beta sin^3 \theta}{(1-\beta cos \theta)^6}
          \\
          \\
          \\
          2 cos \theta sin \theta (1-\beta cos \theta)= 5 \beta sin^3 \theta
          \\
          \\
          \\
          2 cos \theta (1-\beta cos \theta)=5 \beta sin^2 \theta
          \\
          \\
          \\
          2 cos \theta (1-\beta cos \theta)=5 \beta (1-cos^2 \theta)
          \\
          \\
          \\
          2 cos \theta-2 \beta cos^2 \theta=5 \beta-5 \beta cos^2 \theta
          \\
          \\
          \\
          \therefore ~~~ \boxed{
            3 \beta cos^2 \theta+2 cos \theta-5 \beta=0
          } ~~~~ \checkmark
          \\
          \\
          \\
          cos \theta=\dfrac{
            -2 \pm \sqrt{4+60 \beta^2}
          }{
            6 \beta
          }
          =\dfrac{-2 \pm \sqrt{4(1+15 \beta^2)}}{6 \beta}
          =\dfrac{-1 \pm \sqrt{1+15 \beta^2}}{3 \beta}
          \\
          \\
          \\
          \therefore ~~~ \boxed{
            \theta_{max}=Arccos\bigg(\dfrac{-1+\sqrt{1+15 \beta^2}}{3 \beta} \bigg), ~~~ \beta \rightarrow 0
          }
          \\
          \\
          \\
        $ 
        Note that $ \theta_{max}=Arccos\bigg(\dfrac{-1-\sqrt{1+15 \beta^2}}{3 \beta} \bigg)$ is another answer but 
        we want the positive result.
        For the case when the speed $v$ is close to $c$ we have:
        \\
        \\
        $
          \beta \equiv \dfrac{v}{c} \approxeq 1
          \\
          \\
          \text{For lecture we had } \beta=1-\epsilon, \text{assuming } \epsilon \text{is very very small.}
          \\
          \\
          \\
          \dfrac{-1+\sqrt{1+15 (1-\epsilon)^2}}{3 (1-\epsilon)}
          =\dfrac{-1+\sqrt{1+15 (1-\epsilon)^2}}{3} (1+\epsilon)
          =\dfrac{-1+\sqrt{16-30 \epsilon}}{3} (1+\epsilon)
          \\
          \\
          \\
          =(1-\dfrac{5}{4} \epsilon) (1+\epsilon)
          =1-\dfrac{1}{4} \epsilon
          \\
          \\
          \\
          \therefore ~~~ \theta_{max} \approxeq 0 \Longrightarrow cos \theta_{max} \approxeq1-\dfrac{1}{4} \theta^2_{max}
          \\
          \\
          \\
          \Longrightarrow \theta_{max}=\sqrt{\dfrac{\epsilon}{2}}
          \\
          \\
          \\
          \therefore ~~~ \boxed{
            \theta_{max}=\sqrt{\dfrac{(1-\beta)}{2}}
          } ~~~~ \checkmark
          \\
          \\
        $
      }

    \item \textbf{12.9 (20 points)} A Lincoln Continental is twice as long as a $VW$ Beetle, when they
    are at rest. As the Continental overtakes the $VW$, going through a speed trap, a
    (stationary) policeman observes that they both have the same length. The $VW$ is
    going at half the speed of light. How fast is the Lincoln going? (Leave your answer
    as a multiple of $c$.)

      % \textcolor{hwColor}{
      %   \\
      % }

    \item \textbf{12.10 (20 points)} A sailboat is manufactured so that the mast leans at an angle $\bar{\theta}$ with
    respect to the deck. An observer standing on a dock sees the boat go by at speed $v$
    (Fig. $12.14$). What angle does this observer say the mast makes?
    \emph{Note that length contraction can help us understand why synchrotron radiation comes out in the 
    forward direction when going from the particle's frame to the observer frame.}
    \begin{center}
      \includegraphics[height=11cm, width=11cm]{1214.JPG}
    \end{center}

      % \textcolor{hwColor}{
      %   \\
      % }

  \end{enumerate}

\end{document}
