\documentclass[fleqn]{article}
\oddsidemargin 0.0in
\textwidth 6.0in
\thispagestyle{empty}
\usepackage{import}
\usepackage{amsmath}
\usepackage{graphicx}
\usepackage{flexisym}
\usepackage{calligra}
\usepackage{amssymb}
\usepackage{bigints} 
\usepackage[english]{babel}
\usepackage[utf8x]{inputenc}
\usepackage{float}
\usepackage[colorinlistoftodos]{todonotes}


\DeclareMathAlphabet{\mathcalligra}{T1}{calligra}{m}{n}
\DeclareFontShape{T1}{calligra}{m}{n}{<->s*[2.2]callig15}{}
\newcommand{\scriptr}{\mathcalligra{r}\,}
\newcommand{\boldscriptr}{\pmb{\mathcalligra{r}}\,}

\definecolor{hwColor}{HTML}{442020}

\begin{document}

  \begin{titlepage}

    \newcommand{\HRule}{\rule{\linewidth}{0.5mm}}

    \center

    \begin{center}
      \includegraphics[height=11cm, width=11cm]{asu.png}
    \end{center}

    \vline

    \textsc{\LARGE Classical Parts/Field/Matter III}\\[1.5cm]

    \HRule \\[0.5cm]
    { \huge \bfseries Homework Six}\\[0.4cm] 
    \HRule \\[1.0cm]

    \textbf{Behnam Amiri}

    \bigbreak

    \textbf{Prof: Samuel Teitelbaum}

    \bigbreak

    \textbf{{\large \today}\\[2cm]}

    \vfill

  \end{titlepage}

  \begin{enumerate}
    \item \textbf{9.30 (30 points)} Confirm that the energy in the $TE_{mn}$ mode travels at the group velocity. [\emph{Hint:} 
    Find the time averaged Poynting vector $\langle \mathbf{S} \rangle$ and the energy density $\langle u \rangle$ 
    (use Prob. $9.12$ if you wish) Integrate over the cross section of the wave guide to get the energy per unit time 
    and per unit length carried by the wave, and take their ratio.] 
    \emph{You will need to find the transverse components of E and B to get the Poynting vector.}

      % \textcolor{hwColor}{
      %   \\
      % }


    \item \textbf{9.31 (30 points)} Work out the theory of $TM$ modes for a rectangular wave guide. In particular, find the 
    longitudinal electric field, the cutoff frequencies, and the wave and group velocities. Find the ratio of the lowest 
    $TM$ cutoff frequency to the lowest $TE$ cutoff frequency, for a given wave guide. [Caution: What is the lowest $TM$ mode?]
    \emph{Remember that you do not have to find the transverse electric field.  Follow the derivation in $9.5.2$ with the 
    appropriate boundary conditions for TE waves.  You may start with equations $9.179$ and $9.180$ as a given.}

      % \textcolor{hwColor}{
      %   \\
      % }

    \item \textbf{9.40 (40 points)} Consider the \textbf{resonant cavity} produced by closing off the two ends of a rectangular 
    wave guide, at $z=0$ and at $z=d$, making a perfectly conducting empty box. Show that the resonant frequencies for 
    both $TE$ and $TM$ modes are given by
    $$
      \omega_{lmn}=c ~ \pi ~ \sqrt{\bigg(l/d \bigg)^2+\bigg(m/a \bigg)^2+\bigg(n/b \bigg)^2} 
    $$
    for integers $l, m,$ and $n$. Find the associated electric and magnetic fields.
    \emph{We will discuss briefly in class.  Remember, this should almost produce the particle in 
    a finite box wavevectors in three dimensions that you used in quantum mechanics. 
    The energy-momentum relation is different because photons have no mass.}

      % \textcolor{hwColor}{
      %   \\
      % }

  \end{enumerate}

\end{document}
