\documentclass[fleqn]{article}
\oddsidemargin 0.0in
\textwidth 6.0in
\thispagestyle{empty}
\usepackage{import}
\usepackage{amsmath}
\usepackage{graphicx}
\usepackage{flexisym}
\usepackage{calligra}
\usepackage{amssymb}
\usepackage{bigints} 
\usepackage[english]{babel}
\usepackage[utf8x]{inputenc}
\usepackage{float}
\usepackage[colorinlistoftodos]{todonotes}


\DeclareMathAlphabet{\mathcalligra}{T1}{calligra}{m}{n}
\DeclareFontShape{T1}{calligra}{m}{n}{<->s*[2.2]callig15}{}
\newcommand{\scriptr}{\mathcalligra{r}\,}
\newcommand{\boldscriptr}{\pmb{\mathcalligra{r}}\,}

\definecolor{hwColor}{HTML}{442020}

\begin{document}

  \begin{titlepage}

    \newcommand{\HRule}{\rule{\linewidth}{0.5mm}}

    \center

    \begin{center}
      \includegraphics[height=11cm, width=11cm]{asu.png}
    \end{center}

    \vline

    \textsc{\LARGE Classical Parts/Field/Matter III}\\[1.5cm]

    \HRule \\[0.5cm]
    { \huge \bfseries Homework Six}\\[0.4cm] 
    \HRule \\[1.0cm]

    \textbf{Behnam Amiri}

    \bigbreak

    \textbf{Prof: Samuel Teitelbaum}

    \bigbreak

    \textbf{{\large \today}\\[2cm]}

    \vfill

  \end{titlepage}

  \begin{enumerate}
    \item \textbf{9.30 (30 points)} Confirm that the energy in the $TE_{mn}$ mode travels at the group velocity. [\emph{Hint:} 
    Find the time averaged Poynting vector $\langle \mathbf{S} \rangle$ and the energy density $\langle u \rangle$ 
    (use Prob. $9.12$ if you wish) Integrate over the cross section of the wave guide to get the energy per unit time 
    and per unit length carried by the wave, and take their ratio.] 
    \emph{You will need to find the transverse components of E and B to get the Poynting vector.}

      \textcolor{hwColor}{
        \\
        From page 425, 426 and 429 of the textbookw have the following equations and solutions:
        \\
        \\
        $
          \begin{cases}
            \tilde{E}=\tilde{E_0} ~ e^{i(kz-\omega t)}
            \\
            \\
            \tilde{B}=\tilde{B_0} ~ e^{i(kz-\omega t)}
          \end{cases} ~~~~~~~ (9.176)
          \\
          \\
          \\
          \tilde{E_0}=E_x ~ \hat{x}+E_y ~ \hat{y}+E_z ~ \hat{z}, ~~~~~ \tilde{B_0}=B_x ~ \hat{x}+B_y ~ \hat{y}+B_z ~ \hat{z}
          \\
          \\
          \\
          \text{Also, from problem 9.12 we have}
          \\
          \\
          \begin{cases}
            \langle u \rangle=\dfrac{1}{4} ~ Re \bigg( \epsilon_0 ~ \tilde{E}.\tilde{E}^* + \dfrac{1}{\mu_0} \tilde{B}.\tilde{B}^* \bigg) 
            \\
            \\
            \langle S \rangle=\dfrac{1}{2 \mu_0} Re \bigg( \tilde{E} \times \tilde{B}^* \bigg)
          \end{cases}
          \\
          \\
          \\
          B_z=B_0 cos \bigg(m ~ \pi ~ x/a \bigg) ~ cos \bigg(n ~ \pi ~ y/b \bigg) ~~~~~~ (9.186)
          \\
          \\
          \\
          \begin{cases}
            E_x=\dfrac{i}{(\omega / c)^2-k^2} \bigg( k\dfrac{\partial E_z}{\partial x} + \omega \dfrac{\partial B_z}{\partial y} \bigg)
            \\
            \\
            E_y=\dfrac{i}{(\omega / c)^2-k^2} \bigg( k\dfrac{\partial E_z}{\partial y} - \omega \dfrac{\partial B_z}{\partial x} \bigg)
            \\
            \\
            B_x=\dfrac{i}{(\omega / c)^2-k^2} \bigg( k\dfrac{\partial B_z}{\partial x} - \dfrac{\omega}{c^2} \dfrac{\partial E_z}{\partial y} \bigg)
            \\
            \\
            B_y=\dfrac{i}{(\omega / c)^2-k^2} \bigg( k\dfrac{\partial B_z}{\partial y} + \dfrac{\omega}{c^2} \dfrac{\partial E_z}{\partial x} \bigg)
          \end{cases}
          \Longrightarrow
          \begin{cases}
            E_x=-\dfrac{i ~ \omega}{(\omega / c)^2-k^2} \bigg( \dfrac{n ~ \pi}{b} \bigg) ~ B_0 ~ cos\bigg( \dfrac{m ~ \pi ~ x}{a} \bigg) ~ sin\bigg( \dfrac{n ~ \pi ~ y}{b} \bigg)
            \\
            \\
            E_y=\dfrac{i ~ \omega}{(\omega / c)^2-k^2} \bigg( \dfrac{m ~ \pi}{a} \bigg) ~ B_0 ~ sin\bigg( \dfrac{m ~ \pi ~ x}{a} \bigg) ~ cos\bigg( \dfrac{n ~ \pi ~ y}{b} \bigg)
            \\
            \\
            B_x^*=\dfrac{i k~ }{(\omega / c)^2-k^2} ~ \bigg( \dfrac{m ~ \pi}{a} \bigg) ~ B_0 ~ sin\bigg( \dfrac{m ~ \pi ~ x}{a} \bigg) ~ cos\bigg( \dfrac{n ~ \pi ~ y}{b} \bigg)
            \\
            \\
            B_y^*=\dfrac{i ~ k}{(\omega / c)^2-k^2} ~ \bigg( \dfrac{m ~ \pi}{a} \bigg) ~ B_0 ~ sin\bigg( \dfrac{m ~ \pi ~ x}{a} \bigg) ~ cos\bigg( \dfrac{n ~ \pi ~ y}{b} \bigg)
          \end{cases}
          \\
          \\
          \\
          \text{And now we have}
          \\
          \\
          \begin{cases}
            E_z=0
            \\
            \\
            B_z^*=B_0 ~ cos\bigg( \dfrac{m ~ \pi ~ x}{a} \bigg) ~ cos\bigg( \dfrac{n ~ \pi ~ y}{b} \bigg)
          \end{cases}
          \\
          \\
          \\
          \langle S \rangle=\dfrac{1}{2 \mu_0} Re \bigg( \tilde{E} \times \tilde{B}^* \bigg)
          \\
          \\
          =\hat{x} ~ \bigg(\dfrac{1}{2 \mu_0} \dfrac{i ~ \pi ~ \omega ~ B_0^2}{(\omega/c)^2-k^2} ~ \bigg( \dfrac{m}{a} \bigg) ~ sin\bigg( \dfrac{m ~ \pi ~ x}{a} \bigg) ~ cos\bigg( \dfrac{m ~ \pi ~ x}{a} \bigg) ~ cos^2\bigg( \dfrac{n ~ \pi ~ y}{b} \bigg) \bigg)
          \\
          \\
          +\hat{y} ~ \bigg(\dfrac{1}{2 \mu_0} \dfrac{i ~ \pi ~ \omega ~ B_0^2}{(\omega/c)^2-k^2} ~ \bigg( \dfrac{n}{b} \bigg) ~ cos\bigg( \dfrac{n ~ \pi ~ y}{b} \bigg) ~ sin\bigg( \dfrac{n ~ \pi ~ y}{b} \bigg) ~ cos^2\bigg( \dfrac{m ~ \pi ~ x}{a} \bigg) \bigg)
          \\
          \\
          +\hat{z} ~ \bigg(
            \dfrac{1}{2 \mu_0} \dfrac{\omega k ~ \pi^2 ~ B_0^2}{((\omega/c)-k^2)^2} 
            \left[
              cos^2\bigg( \dfrac{m \pi x}{a} \bigg) ~ sin^2\bigg( \dfrac{n ~ \pi ~ y}{b} \bigg) ~ \bigg( \dfrac{n}{b} \bigg)^2
              +cos^2\bigg( \dfrac{n \pi y}{b} \bigg) ~ sin^2\bigg( \dfrac{m ~ \pi ~ x}{a} \bigg) ~ \bigg( \dfrac{m}{a} \bigg)^2
            \right]
          \bigg)
          \\
          \\
          \\
          \boxed{
            \int\limits_{0}^{a} ~ sin^2\bigg( \dfrac{k ~ x}{w} \bigg) ~ dx=\dfrac{w}{2}
          }
          , ~~~~~~~ 
          \boxed{
            \int\limits_{0}^{a} ~ cos^2\bigg( \dfrac{k ~ x}{q} \bigg) ~ dx=\dfrac{q}{2}
          }
          \\
          \\
          \\
          \therefore ~~~ \bigints \langle S \rangle. da=\bigints (
              \dfrac{1}{2 \mu_0} Re \bigg( \tilde{E} \times \tilde{B}^* \bigg)
            \\
            \\
            =\hat{x} ~ \bigg(\dfrac{1}{2 \mu_0} \dfrac{i ~ \pi ~ \omega ~ B_0^2}{(\omega/c)^2-k^2} ~ \bigg( \dfrac{m}{a} \bigg) ~ sin\bigg( \dfrac{m ~ \pi ~ x}{a} \bigg) ~ cos\bigg( \dfrac{m ~ \pi ~ x}{a} \bigg) ~ cos^2\bigg( \dfrac{n ~ \pi ~ y}{b} \bigg) \bigg)
            \\
            \\
            +\hat{y} ~ \bigg(\dfrac{1}{2 \mu_0} \dfrac{i ~ \pi ~ \omega ~ B_0^2}{(\omega/c)^2-k^2} ~ \bigg( \dfrac{n}{b} \bigg) ~ cos\bigg( \dfrac{n ~ \pi ~ y}{b} \bigg) ~ sin\bigg( \dfrac{n ~ \pi ~ y}{b} \bigg) ~ cos^2\bigg( \dfrac{m ~ \pi ~ x}{a} \bigg) \bigg)
            \\
            \\
            +\hat{z} ~ \bigg(
              \dfrac{1}{2 \mu_0} \dfrac{\omega k ~ \pi^2 ~ B_0^2}{((\omega/c)-k^2)^2} 
              \left[
                cos^2\bigg( \dfrac{m \pi x}{a} \bigg) ~ sin^2\bigg( \dfrac{n ~ \pi ~ y}{b} \bigg) ~ \bigg( \dfrac{n}{b} \bigg)^2
                +cos^2\bigg( \dfrac{n \pi y}{b} \bigg) ~ sin^2\bigg( \dfrac{m ~ \pi ~ x}{a} \bigg) ~ \bigg( \dfrac{m}{a} \bigg)^2
              \right]
            \bigg).da
          )
          \\
          \\
          \\
        $
        Solving the above integral with the help of the two (in box) integrals we get the following result:
        \\
        \\
        $
          \therefore ~~~ \boxed{
            \int \langle S \rangle. da=\dfrac{1}{8 \mu_0} \dfrac{a ~ b ~ \omega ~ k ~ \pi^2 ~ B_0^2}{\bigg((\omega/c)^2-k^2 \bigg)^2}
            \times \bigg( \bigg( \dfrac{m}{a} \bigg)^2+\bigg( \dfrac{n}{b} \bigg)^2 \bigg)
          } ~~~~ \checkmark
          \\
          \\
          \\
          \text{On page 429 we have} ~ k=\dfrac{1}{c} ~ \sqrt{\omega^2-\omega^2_{mn}} ~ \text{where} ~ c \pi \sqrt{(m/a)^2+(n/b)^2} 
          \\
          \\
          \\
          \Longrightarrow \boxed{
            \int \langle S \rangle. da=\dfrac{1}{8 \mu_0} \dfrac{\omega ~ k ~ a ~ b ~ B_0^2 ~ c^2}{\omega^2_{mn}}
          } ~~~~ \checkmark
          \\
          \\
          \\
          \\
          \langle u \rangle=\dfrac{1}{4} ~ Re \bigg( \epsilon_0 ~ \tilde{E}.\tilde{E}^* + \dfrac{1}{\mu_0} \tilde{B}.\tilde{B}^* \bigg) 
          =\dfrac{n^2}{b^2} ~ cos^2\bigg( \dfrac{m ~ \pi ~ x}{a} \bigg) ~ sin^2\bigg( \dfrac{n ~ \pi ~ y}{b} \bigg) \dfrac{1}{4} ~ \dfrac{\omega^2 ~ \pi^2 ~ B_0^2}{\bigg( (\omega/c)^2-k^2\bigg)^2}
          \\
          \\
          +\dfrac{m^2}{a^2} ~ cos^2\bigg( \dfrac{n ~ \pi ~ y}{b} \bigg) ~ sin^2\bigg( \dfrac{m ~ \pi ~ x}{a} \bigg) \dfrac{1}{4} ~ \dfrac{\omega^2 ~ \pi^2 ~ B_0^2}{\bigg( (\omega/c)^2-k^2\bigg)^2}
          \\
          \\
          +\dfrac{B_0^2}{4 \mu_0} ~ cos^2\bigg( \dfrac{m ~ \pi ~ x}{a} \bigg) ~ cos^2\bigg( \dfrac{n ~ \pi ~ y}{b} \bigg)
          \\
          \\
          +\dfrac{1}{4 \mu_0 \bigg((\omega/c)^2-k^2 \bigg)^2} \left[
            cos^2\bigg( \dfrac{m \pi x}{a} \bigg) ~ sin^2\bigg( \dfrac{n ~ \pi ~ y}{b} \bigg) ~ \bigg( \dfrac{n}{b} \bigg)^2
            +cos^2\bigg( \dfrac{n \pi y}{b} \bigg) ~ sin^2\bigg( \dfrac{m ~ \pi ~ x}{a} \bigg) ~ \bigg( \dfrac{m}{a} \bigg)^2
          \right]
          \\
          \\
          \\
          \bigints \langle u \rangle. da=
          \bigints (\dfrac{n^2}{b^2} ~ cos^2\bigg( \dfrac{m ~ \pi ~ x}{a} \bigg) ~ sin^2\bigg( \dfrac{n ~ \pi ~ y}{b} \bigg) \dfrac{1}{4} ~ \dfrac{\omega^2 ~ \pi^2 ~ B_0^2}{\bigg( (\omega/c)^2-k^2\bigg)^2}
          \\
          \\
          +\dfrac{m^2}{a^2} ~ cos^2\bigg( \dfrac{n ~ \pi ~ y}{b} \bigg) ~ sin^2\bigg( \dfrac{m ~ \pi ~ x}{a} \bigg) \dfrac{1}{4} ~ \dfrac{\omega^2 ~ \pi^2 ~ B_0^2}{\bigg( (\omega/c)^2-k^2\bigg)^2}
          +\dfrac{B_0^2}{4 \mu_0} ~ cos^2\bigg( \dfrac{m ~ \pi ~ x}{a} \bigg) ~ cos^2\bigg( \dfrac{n ~ \pi ~ y}{b} \bigg)
          \\
          \\
          +\dfrac{1}{4 \mu_0 \bigg((\omega/c)^2-k^2 \bigg)^2} \left[
            cos^2\bigg( \dfrac{m \pi x}{a} \bigg) ~ sin^2\bigg( \dfrac{n ~ \pi ~ y}{b} \bigg) ~ \bigg( \dfrac{n}{b} \bigg)^2
            +cos^2\bigg( \dfrac{n \pi y}{b} \bigg) ~ sin^2\bigg( \dfrac{m ~ \pi ~ x}{a} \bigg) ~ \bigg( \dfrac{m}{a} \bigg)^2
          \right]
          ).da
          \\
          \\
          \\
          \therefore ~~~ \boxed{
            \int \langle u \rangle. da=\dfrac{B_0^2 ~ a ~ b}{16 \mu_0}
            +\dfrac{\epsilon_0 ~ a ~ b}{16} ~ \dfrac{\omega^2 ~ \pi^2 ~ B_0^2}{\left[(\omega/c)^2-k^2\right]^2} \left[\dfrac{n^2}{b^2}+\dfrac{m^2}{a^2}\right]
            +\dfrac{a ~ b}{16 \mu_0} ~ \dfrac{\omega^2 ~ \pi^2 ~ B_0^2}{\left[(\omega/c)^2-k^2\right]^2} \left[\dfrac{n^2}{b^2}+\dfrac{m^2}{a^2}\right]
          } ~~~~ \checkmark
          \\
          \\
          \\
          \text{On page 429 we have} ~ k=\dfrac{1}{c} ~ \sqrt{\omega^2-\omega^2_{mn}} ~ \text{where} ~ c \pi \sqrt{(m/a)^2+(n/b)^2} 
          \\
          \\
          \\
          \Longrightarrow \boxed{
            \int \langle u \rangle. da=\dfrac{1}{8 \mu_0} \dfrac{\omega^2 ~ a ~ b ~ B_0^2}{\omega^2_{mn}}
          } ~~~~ \checkmark
          \\
          \\
          \\
        $
        Now we can take their ratio:
        \\
        \\
        $
          R=\dfrac{
            \dfrac{1}{8 \mu_0} \dfrac{\omega ~ k ~ a ~ b ~ B_0^2 ~ c^2}{\omega^2_{mn}}
          }{
            \dfrac{1}{8 \mu_0} \dfrac{\omega^2 ~ a ~ b ~ B_0^2}{\omega^2_{mn}}
          }=\dfrac{k ~ c^2}{\omega}
          \\
          \\
          \\
          \therefore ~~~ \boxed{
            R=\dfrac{c}{\omega} \sqrt{\omega^2-\omega^2_{mn}}
          } ~~~~ \checkmark
          \\
          \\
          \\
        $
        The above was achieved using equation $k=\dfrac{1}{c} ~ \sqrt{\omega^2-\omega^2_{mn}}$ on page 429.
        \\
        \\
        \\
      }

    \item \textbf{9.31 (30 points)} Work out the theory of $TM$ modes for a rectangular wave guide. In particular, find the 
    longitudinal electric field, the cutoff frequencies, and the wave and group velocities. Find the ratio of the lowest 
    $TM$ cutoff frequency to the lowest $TE$ cutoff frequency, for a given wave guide. [Caution: What is the lowest $TM$ mode?]
    \emph{Remember that you do not have to find the transverse electric field.  Follow the derivation in $9.5.2$ with the 
    appropriate boundary conditions for TE waves.  You may start with equations $9.179$ and $9.180$ as a given.}

      % \textcolor{hwColor}{
      %   \\
      % }

    \item \textbf{9.40 (40 points)} Consider the \textbf{resonant cavity} produced by closing off the two ends of a rectangular 
    wave guide, at $z=0$ and at $z=d$, making a perfectly conducting empty box. Show that the resonant frequencies for 
    both $TE$ and $TM$ modes are given by
    $$
      \omega_{lmn}=c ~ \pi ~ \sqrt{\bigg(l/d \bigg)^2+\bigg(m/a \bigg)^2+\bigg(n/b \bigg)^2} 
    $$
    for integers $l, m,$ and $n$. Find the associated electric and magnetic fields.
    \emph{We will discuss briefly in class.  Remember, this should almost produce the particle in 
    a finite box wavevectors in three dimensions that you used in quantum mechanics. 
    The energy-momentum relation is different because photons have no mass.}

      % \textcolor{hwColor}{
      %   \\
      % }

  \end{enumerate}

\end{document}
