\documentclass[fleqn]{article}
\oddsidemargin 0.0in
\textwidth 6.0in
\thispagestyle{empty}
\usepackage{import}
\usepackage{amsmath}
\usepackage{graphicx}
\usepackage{flexisym}
\usepackage{calligra}
\usepackage{amssymb}
\usepackage{bigints} 
\usepackage[english]{babel}
\usepackage[utf8x]{inputenc}
\usepackage{float}
\usepackage[colorinlistoftodos]{todonotes}


\DeclareMathAlphabet{\mathcalligra}{T1}{calligra}{m}{n}
\DeclareFontShape{T1}{calligra}{m}{n}{<->s*[2.2]callig15}{}
\newcommand{\scriptr}{\mathcalligra{r}\,}
\newcommand{\boldscriptr}{\pmb{\mathcalligra{r}}\,}

\definecolor{hwColor}{HTML}{442020}

\begin{document}

  \begin{titlepage}

    \newcommand{\HRule}{\rule{\linewidth}{0.5mm}}

    \center

    \begin{center}
      \includegraphics[height=11cm, width=11cm]{asu.png}
    \end{center}

    \vline

    \textsc{\LARGE Classical Parts/Field/Matter III}\\[1.5cm]

    \HRule \\[0.5cm]
    { \huge \bfseries Homework 11}\\[0.4cm] 
    \HRule \\[1.0cm]

    \textbf{Behnam Amiri}

    \bigbreak

    \textbf{Prof: Samuel Teitelbaum}

    \bigbreak

    \textbf{{\large \today}\\[2cm]}

    \vfill

  \end{titlepage}

  \begin{enumerate}
    \item \textbf{12.41} Show that the (ordinary) acceleration of a particle of mass m and
    charge $q$, moving at velocity $u$ under the influence of electromagnetic fields $E$ and
    $B$, is given by
    $$
      a=\dfrac{q}{m} \sqrt{1-u^2/c^2} \left[E+u \times B-\dfrac{1}{c^2} u (u.E)\right]
    $$
    [\emph{Hint}: Use Eq. $12.74$.]
    \emph{This is valuable for understanding the forces on charged particles in a synchrotron 
    or laser field. Note the Lorentz force is an ordinary force ($F$, not $K$) and ordinary velocity ($u$, not eta).}

      \textcolor{hwColor}{
        \\
        $
          Eq. ~ 12.74 \longrightarrow F=\dfrac{m}{\sqrt{1-u^2/c^2}} \left[a+\dfrac{u(u.a)}{c^2-u^2}\right]
          =q(E+u \times B)
          \\
          \\
          \\
          \Longrightarrow \dfrac{q}{m} \sqrt{1-u^2/c^2} (E+ u \times B)=a+\dfrac{u(u.a)}{(c^2-u^2)}
          \\
          \\
          \\
          u.a+\dfrac{u^2 (u.a)}{c^2 (1-u^2/c^2)}=\dfrac{q}{m} \sqrt{1-u^2/c^2} \bigg( u.E+u.(u \times B) \bigg)
          \\
          \\
          \\
          \text{Substituting the above value into the below equation gives us:}
          \\
          \\
          \dfrac{u(u.a)}{c^2-u^2}=\dfrac{q \sqrt{1-u^2/c^2}}{m} \dfrac{u(u.E)}{c^2}
          \\
          \\
          \\
          \therefore ~~~ \boxed{
            a=\dfrac{q}{m} \sqrt{1-u^2/c^2} \left[E+u \times B-\dfrac{1}{c^2} u (u.E)\right]
          } ~~~~ \checkmark
          \\
          \\
        $
      }

    \item \textbf{12.48} An electromagnetic plane wave of (angular) frequency $\omega$ is traveling
    in the $x$ direction through the vacuum. It is polarized in the $y$ direction, and the
    amplitude of the electric field is $E_0$.
    \emph{The relativistic doppler shift, done formally.}
    \begin{enumerate}
      \item Write down the electric and magnetic fields, $E(x, y,z, t)$ and $B(x, y,z, t)$. [Be
      sure to define any auxiliary quantities you introduce, in terms of $\omega$, $E_0$, and the
      constants of nature.]

        \textcolor{hwColor}{
          \\
          From page 397, (chapter 9), we have:
          \\
          \\
          $
            E(z, t)=E_0 cos(kz-\omega+\delta) \hat{x}, ~~~ B(z,t)=\dfrac{1}{c} E_0 cos(kz-\omega t+\delta) \hat{y}
            \\
            \\
            \\
            \boxed{
              E(x,y,z, t)=E_0 cos(kx-\omega+0) \hat{y}, ~~~ B(x,y,z,t)=\dfrac{1}{c} E_0 cos(kx-\omega t+0) \hat{z}
            }
            \\
            \\
          $
        }

      \item This same wave is observed from an inertial system $\bar{B}$ moving in the $x$ direction
      with speed $v$ relative to the original system $S$. Find the electric and magnetic
      fields in $\bar{S}$, and express them in terms of the $\bar{S}$ coordinates:  $\bar{E}(\bar{x}, \bar{y}, \bar{z}, \bar{t})$ and
      $\bar{B}(\bar{x}, \bar{y}, \bar{z}, \bar{t})$. [Again, be sure to define any auxiliary quantities you introduce.]

        \textcolor{hwColor}{
          \\
          $
            \alpha \equiv \gamma \bigg( 1-\dfrac{v}{c} \bigg)=\sqrt{\dfrac{
              1-v/c
            }{
              1+v/c
            }}
            \\
            \\
            \begin{cases}
              \bar{E}_x=0
              \\
              \\
              \bar{E}_y=\gamma E_0 \left[cos(k x-\omega t)-\dfrac{v}{c} cos(k x -\omega t)\right]=E_0 \alpha cos(kx-\omega t)
              \\
              \\
              \bar{E}_z=0
            \end{cases}
            \\
            \\
            \\
            \begin{cases}
              \bar{B}_x=0
              \\
              \\
              \bar{B}_y=\gamma E_0 \left[\dfrac{1}{c}cos(k x-\omega t)-\dfrac{v}{c^2} cos(k x -\omega t)\right]=\dfrac{E_0 \alpha}{c} cos(kx-\omega t)
              \\
              \\
              \bar{B}_z=0
            \end{cases}
            \\
            \\
          $
          Now by using the Lorentz transformation we have:
          \\
          \\
          $
            \begin{cases}
              x=\gamma (\bar{x}+v \bar{t})
              \\
              \\
              t=\gamma (\bar{t}+\dfrac{v}{c^2} \bar{x})
            \end{cases} 
            \Longrightarrow \gamma \left[k(\bar{x}+v \bar{t})-\omega (\bar{t}+\dfrac{v}{c^2} \bar{x}) \right]
            =\gamma \left[\bar{x}(k-\dfrac{\omega v}{c^2})-\bar{t}(\omega-kv)\right]=\bar{k} \bar{x}-\bar{\omega} \bar{t}
            \\
            \\
            \\
            \Longrightarrow \bar{k} \bar{x}-\bar{\omega} \bar{t}=kx-\omega t ~~~~ \checkmark
            \\
            \\
            \\
            \begin{cases}
              \bar{k}=\gamma (k-\dfrac{\omega v}{c^2})=\gamma k (1-\dfrac{v}{c})=\alpha k
              \\
              \\
              \bar{\omega}=\gamma \omega (1-\dfrac{v}{c})=\alpha \omega
            \end{cases}
            \\
            \\
            \\
            \therefore ~~~ \boxed{
              \begin{cases}
                \bar{E}(\bar{x}, \bar{y}, \bar{z}, \bar{t})=\bar{E}_0 cos(\bar{k} \bar{x}-\bar{\omega} \bar{t}) ~ \hat{y}
                \\
                \\
                \bar{B}(\bar{x}, \bar{y}, \bar{z}, \bar{t})=\dfrac{\bar{E}_0}{c} cos(\bar{k} \bar{x}-\bar{\omega} \bar{t}) ~ \hat{z}
              \end{cases}
            } ~~~~ \checkmark
            \\
            \\
            \\
            \begin{cases}
              \alpha = \sqrt{\dfrac{1-v/c}{1+v/c}}
              \\
              \\
              \bar{\omega}=\alpha \omega 
              \\
              \\
              \bar{k}=\alpha k
              \\
              \\
              \bar{E}_0=\alpha E_0
            \end{cases}
            \\
            \\
          $
        }

      \item  What is the frequency $\bar{\omega}$ of the wave in $\bar{S}$? Interpret this result. What is the
      wavelength $\bar{\lambda}$ of the wave in $\bar{S}$? From $\bar{\omega}$ and $\bar{\lambda}$, determine the speed of the
      waves in $\bar{S}$. Is it what you expected?

        \textcolor{hwColor}{
          \\
          $
            \begin{cases}
              \bar{\lambda}=\dfrac{2 \pi}{\bar{k}}=\dfrac{2 \pi}{\alpha k}=\dfrac{\lambda}{\alpha}
              \\
              \\
              \bar{v}=\dfrac{\bar{\omega} \bar{\lambda}}{2 \pi}=\dfrac{\omega}{\lambda}=c ~~~~ \checkmark
              \\
              \\
              \bar{\omega}=\omega \sqrt{\dfrac{1-v/c}{1+v/c}}
            \end{cases}
            \\
            \\
          $
          $\bar{\omega}$ is the \textbf{Doppler shift}. We found that $\bar{v}=c$ which is the famous key that
          The key premise to special relativity is that the speed of light is constant in all frames 
          of reference, regardless of their motion. In other words, $c$ is constant in all inertial systems.
          \\
          \\
        }

      \item What is the ratio of the intensity in $\bar{S}$ to the intensity in $\bar{S}$? As a youth, Einstein wondered what an 
      electromagnetic wave would look like if you could run along beside it at the speed of light. What can you tell him 
      about the amplitude, frequency, and intensity of the wave, as $v$ approaches $c$?

        \textcolor{hwColor}{
          \\
          $
            \dfrac{\bar{I}}{I}=\dfrac{\bar{E}^2_0}{E^2_0}=\dfrac{1-v/c}{1+v/c}=\alpha ~~~~ \checkmark
            \\
            \\
          $
          The amplitude, frequency, and intensity of the wave approaches to zero as we run faster and faster. But $c$
          will be still $3 \times 10^{8}$ relative to us.
        }

    \end{enumerate}

  \end{enumerate}

\end{document}
