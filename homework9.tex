\documentclass[fleqn]{article}
\oddsidemargin 0.0in
\textwidth 6.0in
\thispagestyle{empty}
\usepackage{import}
\usepackage{amsmath}
\usepackage{graphicx}
\usepackage{flexisym}
\usepackage{calligra}
\usepackage{amssymb}
\usepackage{bigints} 
\usepackage[english]{babel}
\usepackage[utf8x]{inputenc}
\usepackage{float}
\usepackage[colorinlistoftodos]{todonotes}


\DeclareMathAlphabet{\mathcalligra}{T1}{calligra}{m}{n}
\DeclareFontShape{T1}{calligra}{m}{n}{<->s*[2.2]callig15}{}
\newcommand{\scriptr}{\mathcalligra{r}\,}
\newcommand{\boldscriptr}{\pmb{\mathcalligra{r}}\,}

\definecolor{hwColor}{HTML}{442020}

\begin{document}

  \begin{titlepage}

    \newcommand{\HRule}{\rule{\linewidth}{0.5mm}}

    \center

    \begin{center}
      \includegraphics[height=11cm, width=11cm]{asu.png}
    \end{center}

    \vline

    \textsc{\LARGE Classical Parts/Field/Matter III}\\[1.5cm]

    \HRule \\[0.5cm]
    { \huge \bfseries Homework 9}\\[0.4cm] 
    \HRule \\[1.0cm]

    \textbf{Behnam Amiri}

    \bigbreak

    \textbf{Prof: Samuel Teitelbaum}

    \bigbreak

    \textbf{{\large \today}\\[2cm]}

    \vfill

  \end{titlepage}

  \begin{enumerate}
    \item \textbf{11.14 (30 points)} In Bohr's theory of hydrogen, the electron in its ground state was
    supposed to travel in a circle of radius $5 \times 10^{-11} ~ m$, held in orbit by the Coulomb
    attraction of the proton. According to classical electrodynamics, this electron should
    radiate, and hence spiral in to the nucleus. Show that $v<<c$ for most of the trip
    (so you can use the Larmor formula), and calculate the lifespan of Bohr's atom.
    (Assume each revolution is essentially circular.)

        % \textcolor{hwColor}{
        %   \\
        % }

    \item \textbf{11.17 (30 points)} 
    \begin{enumerate}
      \item A particle of charge $q$ moves in a circle of radius $R$ at a constant speed $v$. To sustain the motion, 
      you must, of course, provide a centripetal force $m v^2/R$; what additional force $(F_e)$ must you exert, 
      in order to counteract the radiation reaction? [It's easiest to express the answer in terms of the instantaneous
      velocity $v$.] What power $(P_e)$ does this extra force deliver? Compare $P_e$ with the
      power radiated (use the Larmor formula).

      \item Repeat part $(a)$ for a particle in simple harmonic motion with amplitude $A$ and
      angular frequency $\omega: w(t) = A cos(\omega t) \hat{z}$. Explain the discrepancy.

        % \textcolor{hwColor}{
        %   \\
        % }

      \item Consider the case of a particle in free fall (constant acceleration g). What is
      the radiation reaction force? What is the power radiated? Comment on these
      results.

        % \textcolor{hwColor}{
        %   \\
        % }

    \end{enumerate}

    \item \textbf{11.18 (40 points)} A point charge $q$, of mass $m$, is attached to a spring of constant $k$.
    At time $t=0$ it is given a kick, so its initial energy is $U_0=\dfrac{1}{2} m v^2_0$. Now it oscillates,
    gradually radiating away this energy.
    \begin{enumerate}
      \item Confirm that the total energy radiated is equal to $U_0$. Assume the radiation
      damping is small, so you can write the equation of motion as
      $$
        \ddot{x}+\gamma \dot{x}+\omega_0 x=0,
      $$
      and the solution as
      $$
        x(t)=\dfrac{v_0}{\omega_0} e^{-\gamma t/2} sin(\omega_0 t),
      $$
      with $\omega_0 \equiv \sqrt{k/m}, ~ \gamma=\omega_0 \tau,$ and $\gamma << \omega_0$ (drop $\gamma^2$ in comparison 
      to $\omega^2_0$, and when you average over a complete cycle, ignore the change in $e^{-\gamma t}$).

        % \textcolor{hwColor}{
        %   \\
        % }

      \item Suppose now we have two such oscillators, and we start them off with identical
      kicks. Regardless of their relative positions and orientations, the total energy
      radiated must be $2 U_0$. But what if they are right on top of each other, so it's
      equivalent to a single oscillator with twice the charge; the Larmor formula says
      that the power radiated is four times as great, suggesting that the total will be
      $4 U_0$. Find the error in this reasoning, and show that the total is actually $2 U_0$, as
      it should be.

        % \textcolor{hwColor}{
        %   \\
        % }

    \end{enumerate}

  \end{enumerate}

\end{document}
