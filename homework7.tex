\documentclass[fleqn]{article}
\oddsidemargin 0.0in
\textwidth 6.0in
\thispagestyle{empty}
\usepackage{import}
\usepackage{amsmath}
\usepackage{graphicx}
\usepackage{flexisym}
\usepackage{calligra}
\usepackage{amssymb}
\usepackage{bigints} 
\usepackage[english]{babel}
\usepackage[utf8x]{inputenc}
\usepackage{float}
\usepackage[colorinlistoftodos]{todonotes}


\DeclareMathAlphabet{\mathcalligra}{T1}{calligra}{m}{n}
\DeclareFontShape{T1}{calligra}{m}{n}{<->s*[2.2]callig15}{}
\newcommand{\scriptr}{\mathcalligra{r}\,}
\newcommand{\boldscriptr}{\pmb{\mathcalligra{r}}\,}

\definecolor{hwColor}{HTML}{442020}

\begin{document}

  \begin{titlepage}

    \newcommand{\HRule}{\rule{\linewidth}{0.5mm}}

    \center

    \begin{center}
      \includegraphics[height=11cm, width=11cm]{asu.png}
    \end{center}

    \vline

    \textsc{\LARGE Classical Parts/Field/Matter III}\\[1.5cm]

    \HRule \\[0.5cm]
    { \huge \bfseries Homework Seven}\\[0.4cm] 
    \HRule \\[1.0cm]

    \textbf{Behnam Amiri}

    \bigbreak

    \textbf{Prof: Samuel Teitelbaum}

    \bigbreak

    \textbf{{\large \today}\\[2cm]}

    \vfill

  \end{titlepage}

  \begin{enumerate}
    \item \textbf{10.7 (20 points)} A time-dependent point charge $q(t)$ at the origin, $\rho(r,t)=q(t) ~ \delta^3(r)$, 
    is fed by a current $J(r,t)=-\dfrac{1}{4 \pi} \bigg(\dfrac{\dot{q}}{r^2}\bigg) ~ \hat{r}$, where $\dot{q} \equiv \dfrac{dq}{dt}$.
    \begin{enumerate}
      \item Check that charge is conserved, by confirming that the continuity equation is
      obeyed.

        \textcolor{hwColor}{
          \\
          In chapter 5 we learned about the continuity equation, $\nabla.J=-\dfrac{\partial \rho}{\partial t}$, (where $J$ is 
          current density and $\rho$ is charge density), which is the precise mathematical statement of local conservation 
          of charge.
          \\
          \\
          $ 
            \begin{cases}
              \nabla.J=-\dfrac{\partial \rho}{\partial t}
              \\
              \\
              J(r,t)=-\dfrac{1}{4 \pi} \bigg(\dfrac{\dot{q}}{r^2}\bigg) ~ \hat{r}
            \end{cases}
            \Longrightarrow
            L.H.S: ~ \nabla.J=\nabla.\bigg(-\dfrac{1}{4 \pi} \bigg(\dfrac{\dot{q}}{r^2}\bigg) ~ \hat{r} \bigg)
            =-\dfrac{\dot{q}}{4 \pi} \left[\nabla.\bigg( \dfrac{\hat{r}}{r^2} \bigg)\right]
            \\
            \\
            \\
            \text{In chapter 1, page 50, equation 1.99, we have} ~~ \boxed{\nabla.\bigg( \dfrac{\hat{r}}{r^2} \bigg)=4 \pi \delta^3(r)}. \text{Therefore:}
            \\
            \\
            \\
            \nabla.J=-\dfrac{\dot{q}}{4 \pi} \left[ \nabla.\bigg( \dfrac{\hat{r}}{r^2} \bigg)\right]
            =-\dfrac{\dot{q}}{4 \pi} \times 4 \pi \delta^3(r)
            =-\dot{q} \delta^3(r)
            =-\dfrac{dq}{dt} \delta^3(r) ~~~~~ \textbf{(A)}
            \\
            \\
          $
          A point charge is confined to a single point in space. Let's call it as $q(\overrightarrow{r})$. This means that 
          the charge has a magnitude only at $\overrightarrow{r}$ , and at all other points it is zero. So, a point charge 
          has a non-zero magnitude at the point it occupies. This particular property of the charge density of a point 
          charge is exactly identical to the definition of the Dirac-delta function. Therefore, it seems quite comfortable 
          that we could use this function to represent the volume charge density of a point charge.
          At the origin, the charge is non-zero, but the inverse volume blows up. So, we substitute the Dirac-delta 
          function in the place of inverse volume as
          $$
            \rho=q ~ \delta^3(\overrightarrow{r})
          $$
          hence, we can rewrite \textbf{(A)} as 
          \\
          \\
          $
            L.H.S: ~ -\dfrac{d}{dt} \left[q ~ \delta^3(r)\right]=-\dfrac{d}{dt} \rho 
            \\
            \\
            \\
            \therefore ~~~ \boxed{L.H.S=-\dfrac{d}{dt} \rho}=R.H.S ~~~~ \checkmark
            \\
            \\
            \\
          $
          With the above result the we can state that the charge is conserved, since the continuity equation holds.
          \\
          \\
        }

      \pagebreak

      \item Find the scalar and vector potentials in the Coulomb gauge. If you get stuck, try
      working on (c) first.

        \textcolor{hwColor}{
          \\
          From page 441 of the textbook we have
          $
            \\
            \\
            V(r,t)=\dfrac{1}{4 \pi \epsilon_0} \bigint \dfrac{\rho(r^', t)}{\scriptr} d\tau^'
            \\
            \\
            \\
            \text{Based on the part (a) of this problem set, we have the following:}
            \\
            \\
            V(r,t)=\dfrac{q}{4 \pi \epsilon_0} \bigint \dfrac{\delta^3(r^')}{\scriptr} d\tau^'
            \\
            \\
            \\
            \therefore ~~~ \boxed{
              V(r,t)=\dfrac{q}{4 \pi \epsilon_0 r}
            } ~~~~ \checkmark
            \\
            \\
            \\
            \\
            \therefore ~~~ \begin{cases}
              \nabla \times A=0
              \\
              \\
              \nabla.A=0
            \end{cases} \Longrightarrow A=0
            \\
            \\
          $
          Note that one solution is zero. but it is not the only solution. The other solutions correspond to freely 
          propagating electromagnetic waves, and no symmetry argument eliminates them ($\overrightarrow{B} \neq 0$ for these solutions). 
          But in any case these electromagnetic waves are not caused by the changing $\rho$ so it seems this problem wants 
          you to take $\overrightarrow{A}=0$. You should be able to verify Maxwell's equations whether you take it to vanish or not.
          \\
          \\
        }

      \item Find the fields, and check that they satisfy all of Maxwell’s equations. (\emph{P. R. Berman, Am. J. Phys. 76 48 (2008).})

        \textcolor{hwColor}{
          \\
          $
            E=-\nabla V-\dfrac{\partial A}{\partial t}
            \\
            \\
            \\
            E=\dfrac{1}{4 \pi \epsilon_0} \dfrac{q(t)}{r^2} \hat{r} ~ B=0 ~~~~ \checkmark
            \\
            \\
            \\
            \nabla \times E=-\dfrac{\partial B}{\partial t}=0 ~~~~ \checkmark
            \\
            \\
            \\
            \nabla.E=\dfrac{1}{4 \pi \epsilon_0} \bigg( q \bigg) ~ \nabla.\bigg( \dfrac{q ~ \delta^3(r)}{\epsilon_0} \bigg) 
            \Longrightarrow \nabla.E=\dfrac{\rho}{\epsilon_0} ~~~~ \checkmark
            \\
            \\
            \\
            \nabla \times B=0
            \\
            \\
            \\
            \nabla \times B=\mu_0 J+\mu_0 \epsilon_0 \dfrac{\partial E}{\partial t}
            \Longrightarrow \mu_0 J+\mu_0 \epsilon_0 \dfrac{\partial E}{\partial t}=\mu_0 \left[
              -\dfrac{1}{4 \pi} \dfrac{\dot{q}}{r^2} \hat{r}
            \right]+\mu_0 \epsilon_0 \left[
              \dfrac{\dot{q}}{4 \pi \epsilon_0 r^2} \hat{r}
            \right]=0 ~~~~ \checkmark
            \\
            \\
          $
        }

    \end{enumerate}


    \item \textbf{10.10 (30 points)} Confirm that the retarded potentials satisfy the Lorenz gauge condition. [
      \emph{Hint:} First show that
      $$
        \nabla.\bigg(\dfrac{J}{\scriptr}\bigg)=\dfrac{1}{\scriptr} \bigg(\nabla.J\bigg)+\dfrac{1}{\scriptr} \bigg(\nabla^'.J\bigg)-\nabla^'.\bigg(\dfrac{J}{\scriptr}\bigg),
      $$
      where $\nabla$ denotes derivatives with respect to $r$, and $\nabla^'$ denotes derivatives with respect to
      $r^'$. Next, noting that $J(r^', t-\scriptr /c)$ depends on $r^'$ both explicitly and through $\scriptr$,
      whereas it depends on $r$ only through $\scriptr$, confirm that
      $$
        \nabla.J=-\dfrac{1}{c} \dot{J}.\bigg(\nabla ~ \scriptr \bigg), ~~~ \nabla^'.J=-\dot{\rho}-\dfrac{1}{c} \dot{J}.\bigg(\nabla^' ~ \scriptr\bigg).
      $$
      Use this to calculate the divergence of $A$ (Eq. 10.26).
      ]

        \textcolor{hwColor}{
          \\
          Let's start off with equation 10.26
          $$
            V(r,t)=\dfrac{1}{4 \pi \epsilon_0} \bigints \dfrac{\rho(r^', t_r)}{\scriptr} d\tau^', ~~ A(r,t)=\dfrac{\mu_0}{4 \pi} \bigints \dfrac{J(r^', t_r)}{\scriptr} d\tau^'
          $$
          From the \textbf{Vector Identities} section of the textbook, product rule 5 states
          $$
            \nabla.\bigg( f A\bigg)=f \bigg( \nabla.A \bigg)+A.\bigg( \nabla f\bigg),
          $$
          therefore, we have:
          \\
          \\
          $
            \begin{cases}
              \nabla.\bigg( \dfrac{J}{\scriptr}\bigg)=\dfrac{1}{\scriptr} \bigg( \nabla.J\bigg)+J.\bigg( \nabla \dfrac{1}{\scriptr}\bigg)
              \\
              \\
              \nabla^'.\bigg( \dfrac{J}{\scriptr} \bigg)=\dfrac{1}{\scriptr} \bigg( \nabla^'.J \bigg)+J.\bigg( \nabla^' \dfrac{1}{\scriptr} \bigg)
            \end{cases}
            \\
            \\
            \\
            \nabla.\bigg( \dfrac{J}{\scriptr} \bigg)=\dfrac{1}{\scriptr} \bigg( \nabla.J \bigg)-J.\bigg( \nabla^' \dfrac{1}{r} \bigg)
            =\dfrac{1}{r} \bigg( \nabla.J \bigg)+\dfrac{1}{\scriptr} \bigg( \nabla^'.J \bigg)-\nabla^'.\bigg( \dfrac{J}{\scriptr} \bigg)
            \\
            \\
            \\
            \nabla.J=\dfrac{\partial J_x}{\partial x}+\dfrac{\partial J_y}{\partial y}+\dfrac{\partial J_z}{\partial z}
            =\dfrac{\partial J_x}{\partial t_r} \dfrac{\partial t_r}{\partial x}
            +\dfrac{\partial J_y}{\partial t_r} \dfrac{\partial t_r}{\partial y}
            +\dfrac{\partial J_z}{\partial t_r} \dfrac{\partial t_r}{\partial z}
            \\
            \\
            \\
            \begin{cases}
              \dfrac{\partial t_r}{\partial x}=-\dfrac{1}{c} \dfrac{\partial r}{\partial x}
              \\
              \\
              \dfrac{\partial t_r}{\partial y}=-\dfrac{1}{c} \dfrac{\partial r}{\partial y}
              \\
              \\
              \dfrac{\partial t_r}{\partial z}=-\dfrac{1}{c} \dfrac{\partial r}{\partial z}
            \end{cases}
            \\
            \\
            \\
            \\
            \nabla.J=\dfrac{\partial J_x}{\partial t_r} \bigg( -\dfrac{1}{c} \dfrac{\partial r}{\partial x} \bigg)
            +\dfrac{\partial J_y}{\partial t_r} \bigg( -\dfrac{1}{c} \dfrac{\partial r}{\partial y} \bigg)
            +\dfrac{\partial J_z}{\partial t_r} \bigg( -\dfrac{1}{c} \dfrac{\partial r}{\partial z} \bigg)
            \\
            \\
            \\
            \nabla.J=-\dfrac{1}{c} \left[
              \dfrac{\partial J_x}{\partial t_r} \dfrac{\partial \scriptr}{\partial x}
            +\dfrac{\partial J_y}{\partial t_r} \dfrac{\partial \scriptr}{\partial y}
            +\dfrac{\partial J_z}{\partial t_r} \dfrac{\partial \scriptr}{\partial z}
            \right]
            \\
            \\
            \\
            \therefore ~~~ \boxed{
              \nabla.J=-\dfrac{1}{c} \dfrac{\partial J}{\partial t_r}.\bigg( \nabla \scriptr \bigg)
            } ~~~~ \checkmark
            \\
            \\
            \\
            \\
            \\
            \nabla^'.J=\left[
              \dfrac{\partial J^'_x}{\partial x^'}+\dfrac{\partial J^'_x}{\partial t_r} \dfrac{\partial t_r}{\partial x^'}
            \right]
            +
            \left[
              \dfrac{\partial J^'_y}{\partial y^'}+\dfrac{\partial J^'_y}{\partial t_r} \dfrac{\partial t_r}{\partial y^'}
            \right]
            +
            \left[
              \dfrac{\partial J^'_z}{\partial z^'}+\dfrac{\partial J^'_z}{\partial t_r} \dfrac{\partial t_r}{\partial z^'}
            \right]
            \\
            \\
            \\
            \nabla^'.J=\left[
              \dfrac{\partial J^'_x}{\partial x^'}+\dfrac{\partial J^'_y}{\partial y^'}+\dfrac{\partial J^'_z}{\partial z^'}
            \right]
            -\dfrac{1}{c} \left[
              \dfrac{\partial J^'_x}{\partial t_r} \dfrac{\partial \scriptr}{\partial x^'}
              +\dfrac{\partial J^'_y}{\partial t_r} \dfrac{\partial \scriptr}{\partial y^'}
              +\dfrac{\partial J^'_z}{\partial t_r} \dfrac{\partial \scriptr}{\partial z^'}
            \right]
            \\
            \\
            \\
            \therefore ~~~ \boxed{
              \nabla^'.J=-\dfrac{\partial \rho}{\partial t}-\dfrac{1}{c} \dfrac{\partial J}{\partial t_r}.\bigg( \nabla^' \scriptr \bigg)
            } ~~~~ \checkmark
            \\
            \\
            \\
            \\
            \\
            \nabla.\bigg( \dfrac{J}{\scriptr} \bigg)=\dfrac{1}{\scriptr} \bigg( -\dfrac{1}{c} \dfrac{\partial J}{\partial t_r}.\bigg( \nabla^' \scriptr \bigg) \bigg)
            +\dfrac{1}{\scriptr} \bigg( -\dfrac{\partial \rho}{\partial t}-\dfrac{1}{c} \dfrac{\partial J}{\partial t_r}.\bigg( \nabla^' \scriptr \bigg) \bigg)
            -\nabla.\bigg( \dfrac{J}{\scriptr} \bigg)
            \\
            \\
            \\
            \therefore ~~~ \boxed{
              \nabla.\bigg( \dfrac{J}{\scriptr} \bigg)=-\dfrac{1}{\scriptr} \dfrac{\partial \rho}{\partial t}-\nabla^'.\bigg( \dfrac{J}{\scriptr} \bigg)
              ~~~~ \Longleftrightarrow \nabla^' \scriptr=-\nabla \scriptr
            } ~~~~ \checkmark
            \\
            \\
            \\
            \nabla.A=\dfrac{\mu_0}{4 \pi} \left[
              -\dfrac{\partial }{\partial t} \bigints \dfrac{\rho}{\scriptr} d\tau-\bigints\nabla^'.\bigg( \dfrac{J}{\scriptr} \bigg)
            \right]
            \\
            \\
            \\
            \nabla.A=-\mu_0 \epsilon_0 \dfrac{\partial }{\partial t} \left[
              \dfrac{1}{4 \pi \epsilon_0} \bigints \dfrac{\rho}{\scriptr} d\tau
            \right]-\dfrac{\mu_0}{4 \pi} \oint \dfrac{J}{\scriptr} da ~~~~~ \Longleftrightarrow J=0
            \\
            \\
            \\
            \therefore ~~~ \boxed{
              \nabla.A=-\mu_0 \epsilon_0 \dfrac{\partial V}{\partial t}
            } ~~~~ \checkmark
            \\
            \\
          $
        }

    \item \textbf{10.11 (30 points)}
      \begin{enumerate}
        \item Suppose the wire in Ex. 10.2 carries a linearly increasing current
        $$
          I(t)=kt,
        $$
        for $t > 0$. Find the electric and magnetic fields generated.

          \textcolor{hwColor}{
            \\
            $
              A(r,t)=\bigg( \dfrac{\mu_0}{4 \pi} ~ \hat{z}\bigg) \bigints\limits_{0}^{\sqrt{(ct)^2-r^2}} \dfrac{k \bigg(t-\dfrac{\sqrt{r^2+z^2}}{c} \bigg)}{\sqrt{r^2+z^2}} dz
              \\
              \\
              \\
              =\bigg( \dfrac{\mu_0 k}{2 \pi} \bigg) \bigg( \bigints\limits_{0}^{\sqrt{(ct)^2-r^2}} \dfrac{t}{\sqrt{r^2+z^2}} dz -\dfrac{1}{c} \bigints\limits_{0}^{\sqrt{(ct)^2-r^2}}  dz    \bigg) \hat{z}
              \\
              \\
              \\
              \therefore ~~~ \boxed{
                A(r,t)=\bigg( \dfrac{\mu_0 k}{2 \pi} \bigg)     
                \left[
                  t ~ ln\bigg( \dfrac{ct+\sqrt{c^2 t^2-r^2}}{r}\bigg)
                  -\dfrac{1}{c} \sqrt{c^2 t^2-r^2}
                \right]
                \hat{z}
              } ~~~~ \checkmark
              \\
              \\
              \\
              E(r,t)=-\dfrac{\partial A}{\partial t}
              =-\dfrac{\partial}{\partial t} \left[
                \bigg( \dfrac{\mu_0 k}{2 \pi} \bigg)     
                \left[
                  t ~ ln\bigg( \dfrac{ct+\sqrt{c^2 t^2-r^2}}{r}\bigg)
                  -\dfrac{1}{c} \sqrt{c^2 t^2-r^2}
                \right]
                \hat{z}
              \right]
              \\
              \\
              \\
              E(r,t)=-\bigg( \dfrac{\mu_0 k}{2 \pi} \bigg) 
              \left[
                ln\bigg( \dfrac{ct+\sqrt{c^2 t^2-r^2}}{r}\bigg)
                +\dfrac{t}{r} \bigg( \dfrac{r}{ct+\sqrt{c^2 t^2-r^2}} \bigg) \bigg( c+\dfrac{1}{2} \dfrac{2c^2 t}{\sqrt{c^2 t^2-r^2}} \bigg)
                -\dfrac{1}{2c} \dfrac{2c^2t}{\sqrt{c^2 t^2-r^2}}
              \right]
              \hat{z}
              \\
              \\
              \\
              \\
              \therefore ~~~ \boxed{
                E(r,t)=-\dfrac{\mu_0 k}{2 \pi} ~ ln\bigg( \dfrac{ct+\sqrt{c^2t^2-r^2}}{r} \bigg) \hat{z}
              } ~~~~ \checkmark
              \\
              \\
              \\
              \\
              B(r,t)=-\dfrac{\partial A_z}{\partial r} \hat{\Phi}
              =-\dfrac{\partial}{\partial r} \left[
                \bigg( \dfrac{\mu_0 k}{2 \pi} \bigg)     
                \left[
                  t ~ ln\bigg( \dfrac{ct+\sqrt{c^2 t^2-r^2}}{r}\bigg)
                  -\dfrac{1}{c} \sqrt{c^2 t^2-r^2}
                \right]
              \right]
              \hat{\Phi}
              \\
              \\
              \\
              =-\dfrac{\mu_0 k}{2 \pi} \left[
                \bigg( \dfrac{tr}{ct+\sqrt{c^2t^2-r^2}} \bigg)
                \dfrac{
                  \bigg( -\dfrac{r^2}{\sqrt{c^2 t^2-r^2}}\bigg)-ct-\sqrt{c^2 t^2-r^2}
                }{r^2}
                +\dfrac{1}{c} \dfrac{r}{\sqrt{c^2t^2-r^2}}
              \right] ~ \hat{\Phi}
              \\
              \\
              \\
              =-\dfrac{\mu_0 k}{2 \pi r c} \dfrac{-c^2 t^2-r^2}{\sqrt{c^2 t^2-r^2}} ~ \hat{\Phi}
              \\
              \\
              \\
              \\
              \therefore ~~~ \boxed{
                B(r,t)=\dfrac{\mu_0 k}{2 \pi r c} \sqrt{c^2 t^2-r^2} ~ \hat{\Phi}
              } ~~~~ \checkmark
              \\
              \\
            $
          }

        \item Do the same for the case of a sudden burst of current:
        $$
          I(t)=q_0 \delta(t).
        $$

          \textcolor{hwColor}{
            \\
            $
              A(r,t)=\dfrac{\mu_0}{4 \pi} \bigints\limits_{- \infty}^{+ \infty} \dfrac{q_0 \delta(t-\scriptr/c)}{\scriptr} ~ dz ~ \hat{z}
              =\dfrac{\mu_0}{2 \pi} \bigints\limits_{0}^{+ \infty} \dfrac{q_0 \delta(t-\scriptr/c)}{\scriptr} ~ dz ~ \hat{z}
              \\
              \\
              \\
              \begin{cases}
                \scriptr=\sqrt{r^2+z^2}
                \\
                \\
                or
                \\
                \\
                z=\sqrt{\scriptr^2-r^2}
              \end{cases} \Longrightarrow \begin{cases}
                dz=\dfrac{1}{2} ~ \dfrac{2 \scriptr ~ d\scriptr}{\sqrt{\scriptr-r^2 }}
                \\
                \\
                z=\infty \longrightarrow \scriptr=\infty
                \\
                \\
                z=0 \longrightarrow \scriptr=r
              \end{cases}
              \\
              \\
              \\
              \therefore ~~~ A(r,t)=\dfrac{\mu_0 q_0}{2 \pi} \bigints\limits_{r}^{+ \infty} 
              \dfrac{\delta(t-\scriptr/c)}{\scriptr} \dfrac{\scriptr}{\sqrt{\scriptr^2-r^2}} ~ d\scriptr ~~ \hat{z}
              \\
              \\
              \\
              \text{Knowing that } \delta(t-\scriptr/c)=c ~ \delta(\scriptr-ct)
              \\
              \\
              \\
              \\
              \therefore ~~~ \boxed{
                \begin{cases}
                  A(r,t)=\dfrac{\mu_0 q_0 c}{2 \pi} \dfrac{1}{\sqrt{c^2t^2-r^2}} ~ \hat{z} ~~~~ r > ct
                  \\
                  \\
                  \\
                  E(r,t)=-\dfrac{\partial A}{\partial t}=\dfrac{\mu_0 q_0 c^3 t}{2 \pi \left[c^2 t^2-r^2\right]^{3/2}} ~ \hat{z} ~~~~ r/c > t
                  \\
                  \\
                  \\
                  B(r,t)=-\dfrac{\partial A_z}{\partial \Phi}=-\dfrac{\mu_0 q_0 c r}{2 \pi \left[c^2 t^2-r^2\right]^{3/2}} ~ \hat{\Phi} ~~~~ r/c > t
                \end{cases}
              } ~~~~ \checkmark
              \\
              \\
            $
          }
          
      \end{enumerate}

    \item \textbf{10.13 (20 points)} Suppose $J(r)$ is constant in time, so (Prob. 7.60) $\rho(r,t)=\rho(r,0)+\dot{\rho}(r,0)t$.
    Show that 
    $$
      E(r,t)=\dfrac{1}{4 \pi \epsilon_0} \int \dfrac{\rho(r^', t)}{\scriptr^2} ~ \hat{\scriptr} ~ d \tau^'
    $$
    that is, Coulomb’s law holds, with the charge density evaluated at the non-retarded time.

      % \textcolor{hwColor}{
      %   \\
      % }

      % \bigg( \bigg)

  \end{enumerate}

\end{document}
