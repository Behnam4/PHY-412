\documentclass[fleqn]{article}
\oddsidemargin 0.0in
\textwidth 6.0in
\thispagestyle{empty}
\usepackage{import}
\usepackage{amsmath}
\usepackage{graphicx}
\usepackage{flexisym}
\usepackage{calligra}
\usepackage{amssymb}
\usepackage{bigints} 
\usepackage[english]{babel}
\usepackage[utf8x]{inputenc}
\usepackage{float}
\usepackage[colorinlistoftodos]{todonotes}


\DeclareMathAlphabet{\mathcalligra}{T1}{calligra}{m}{n}
\DeclareFontShape{T1}{calligra}{m}{n}{<->s*[2.2]callig15}{}
\newcommand{\scriptr}{\mathcalligra{r}\,}
\newcommand{\boldscriptr}{\pmb{\mathcalligra{r}}\,}

\definecolor{hwColor}{HTML}{442020}

\begin{document}

  \begin{titlepage}

    \newcommand{\HRule}{\rule{\linewidth}{0.5mm}}

    \center

    \begin{center}
      \includegraphics[height=11cm, width=11cm]{asu.png}
    \end{center}

    \vline

    \textsc{\LARGE Classical Parts/Field/Matter III}\\[1.5cm]

    \HRule \\[0.5cm]
    { \huge \bfseries Homework Seven}\\[0.4cm] 
    \HRule \\[1.0cm]

    \textbf{Behnam Amiri}

    \bigbreak

    \textbf{Prof: Samuel Teitelbaum}

    \bigbreak

    \textbf{{\large \today}\\[2cm]}

    \vfill

  \end{titlepage}

  \begin{enumerate}
    \item \textbf{10.7 (20 points)} A time-dependent point charge $q(t)$ at the origin, $\rho(r,t)=q(t) ~ \delta^3(r)$, 
    is fed by a current $J(r,t)=-\dfrac{1}{4 \pi} \bigg(\dfrac{\dot{q}}{r^2}\bigg) ~ \hat{r}$, where $\dot{q} \equiv \dfrac{dq}{dt}$.
    \begin{enumerate}
      \item Check that charge is conserved, by confirming that the continuity equation is
      obeyed.

        % \textcolor{hwColor}{
        %   \\
        % }

      \item Find the scalar and vector potentials in the Coulomb gauge. If you get stuck, try
      working on (c) first.

        % \textcolor{hwColor}{
        %   \\
        % }

      \item Find the fields, and check that they satisfy all of Maxwell’s equations. (\emph{P. R. Berman, Am. J. Phys. 76 48 (2008).})

        % \textcolor{hwColor}{
        %   \\
        % }

    \end{enumerate}


    \item \textbf{10.10 (30 points)} Confirm that the retarded potentials satisfy the Lorenz gauge condition. [
      \emph{Hint:} First show that
      $$
        \nabla.\bigg(\dfrac{J}{\scriptr}\bigg)=\dfrac{1}{\scriptr} \bigg(\nabla.J\bigg)+\dfrac{1}{\scriptr} \bigg(\nabla^'.J\bigg)-\nabla^'.\bigg(\dfrac{J}{\scriptr}\bigg),
      $$
      where $\nabla$ denotes derivatives with respect to $r$, and $\nabla^'$ denotes derivatives with respect to
      $r^'$. Next, noting that $J(r^', t-\scriptr /c)$ depends on $r^'$ both explicitly and through $\scriptr$,
      whereas it depends on $r$ only through $\scriptr$, confirm that
      $$
        \nabla.J=-\dfrac{1}{c} \dot{J}.\bigg(\nabla ~ \scriptr \bigg), ~~~ \nabla^'.J=-\dot{\rho}-\dfrac{1}{c} \dot{J}.\bigg(\nabla^' ~ \scriptr\bigg).
      $$
      Use this to calculate the divergence of $A$ (Eq. 10.26).
      ]

        % \textcolor{hwColor}{
        %   \\
        % }

    \item \textbf{10.11 (30 points)}
      \begin{enumerate}
        \item Suppose the wire in Ex. 10.2 carries a linearly increasing current
        $$
          I(t)=kt,
        $$
        for $t > 0$. Find the electric and magnetic fields generated.

          % \textcolor{hwColor}{
          %   \\
          % }

        \item Do the same for the case of a sudden burst of current:
        $$
          I(t)=q_0 \delta(t).
        $$

          % \textcolor{hwColor}{
          %   \\
          % }
          
      \end{enumerate}


    \item \textbf{10.13 (20 points)} Suppose $J(r)$ is constant in time, so (Prob. 7.60) $\rho(r,t)=\rho(r,0)+\dot{\rho}(r,0)t$.
    Show that 
    $$
      E(r,t)=\dfrac{1}{4 \pi \epsilon_0} \int \dfrac{\rho(r^', t)}{\scriptr^2} ~ \hat{\scriptr} ~ d \tau^'
    $$
    that is, Coulomb’s law holds, with the charge density evaluated at the non-retarded time.

      % \textcolor{hwColor}{
      %   \\
      % }

  \end{enumerate}

\end{document}
