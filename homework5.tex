\documentclass[fleqn]{article}
\oddsidemargin 0.0in
\textwidth 6.0in
\thispagestyle{empty}
\usepackage{import}
\usepackage{amsmath}
\usepackage{graphicx}
\usepackage{flexisym}
\usepackage{calligra}
\usepackage{amssymb}
\usepackage{bigints} 
\usepackage[english]{babel}
\usepackage[utf8x]{inputenc}
\usepackage{float}
\usepackage[colorinlistoftodos]{todonotes}


\DeclareMathAlphabet{\mathcalligra}{T1}{calligra}{m}{n}
\DeclareFontShape{T1}{calligra}{m}{n}{<->s*[2.2]callig15}{}
\newcommand{\scriptr}{\mathcalligra{r}\,}
\newcommand{\boldscriptr}{\pmb{\mathcalligra{r}}\,}

\definecolor{hwColor}{HTML}{442020}

\begin{document}

  \begin{titlepage}

    \newcommand{\HRule}{\rule{\linewidth}{0.5mm}}

    \center

    \begin{center}
      \includegraphics[height=11cm, width=11cm]{asu.png}
    \end{center}

    \vline

    \textsc{\LARGE Classical Parts/Field/Matter III}\\[1.5cm]

    \HRule \\[0.5cm]
    { \huge \bfseries Homework Five}\\[0.4cm] 
    \HRule \\[1.0cm]

    \textbf{Behnam Amiri}

    \bigbreak

    \textbf{Prof: Samuel Teitelbaum}

    \bigbreak

    \textbf{{\large \today}\\[2cm]}

    \vfill

  \end{titlepage}

  \begin{enumerate}
    \item \textbf{9.21 (10 points)}
    \begin{enumerate}
      \item Calculate the (time-averaged) energy density of an electromagnetic plane wave in a conducting medium (Eq. 9.138). 
      Show that the magnetic contribution always dominates. 
      $\left[\emph{Answer:} \left(k^2/2 \mu ~ \omega^2\right)E^2_0 ~ e^{-2 \kappa ~ z}\right]$

        % \textcolor{hwColor}{
        %   \\
        % }


      \item Show that the intensity is $\left(k/2 \mu ~ \omega\right)E^2_0 ~ e^{-2 \kappa ~ z}$.
  
        % \textcolor{hwColor}{
        %   \\
        % }

    \end{enumerate}
    

    \item \textbf{9.25 (20 points)} Find the width of the anomalous dispersion region for the case of a single resonance at 
    frequency $\omega_0$. Assume $\gamma << \omega_0$. Show that the index of refraction assumes its maximum and minimum values at points where the absorption
    coefficient is at half-maximum.

        % \textcolor{hwColor}{
        %   \\
        % }
    
    \item \textbf{9.26 (10 points)} Starting with Eq. 9.170, calculate the group velocity, assuming there
    is only one resonance, at $\omega_0$.  Use a computer to graph $y \equiv \dfrac{v_g}{c}$ as a function of 
    $x \equiv \left(\dfrac{\omega}{\omega_0}\right)$, from $x=0$ to $2$. (a) for $\gamma=0$, and (b) $\gamma=\left(0.1\right) \omega_0$. 
    Let $\dfrac{N ~ q^2}{2 m ~ \epsilon_0 ~ \omega^2_0}=0.003$. Note that the group velocity can exceed $c$.

        % \textcolor{hwColor}{
        %   \\
        % }

    
    \item \textbf{1 (40 points)} The Drude model describes the interactions of n electrons with the electric field via a drag term. One 
    can think of this as the harmonic oscillator model used in Griffith’s to describe, but with no resonant frequency because there is no 
    restoring force on the electron position (because in a conductor electrons can move somewhat freely).

    To begin, we will reconsider eq. $(9.154)$ in Griffiths with $\omega_0=0$. Consider the motion of the electrons
    to be described by the equation
    $$
      m ~ \partial^2_t ~ x+m ~ \gamma ~ \partial_t x=e ~ \tilde{E}(t)
    $$

    \begin{enumerate}
      \item Derive the complex-valued $\tilde{\epsilon}(\omega)$ for this model using the method used in class or in the textbook.
      Using the relationship between $\tilde{\sigma}(\omega)$ and $\tilde{\epsilon}(\omega)$, show that the 
      Frequency-dependent conductivity for a metal in this model is
      $$
        \sigma(\omega)=\dfrac{\sigma_0 ~ \gamma}{1-i ~ \omega ~ \gamma}
      $$

        % \textcolor{hwColor}{
        %   \\
        % }

      \item Show that the (real part) of the conductivity $\sigma$ in the limit of small frequency $\omega$ is given by
      $$
        \sigma_0=\dfrac{\omega^2_p}{\gamma}
      $$
      Where $\omega_p$ is the plasma frequency $\omega_p=\sqrt{\dfrac{N ~ e^2}{m}}$. Clearly state what ”small” means in this
      context.

        % \textcolor{hwColor}{
        %   \\
        % }
      
      \item We know that a typical conductivity for a noble metal, e.g Gold, is $4 \times 10^7 ~ \left(\Omega m\right)^{-1}$, and the
      plasma frequency is $\omega_p= 10^{15} ~ rad/s$. What is the relaxation time $\tau=\dfrac{1}{\gamma}$ for gold? What
      is the physical interpretation of this relaxation time (write 1-2 sentences)?. Think about the
      equation of motion you used in (a).

        % \textcolor{hwColor}{
        %   \\
        % }
      
      \item Graph the real and imaginary parts of $\tilde{n}(\omega)$, $\tilde{\epsilon}(\omega)$, and $\tilde{\sigma}(\omega)$. Based on these graphs, over what
      frequency range is a constant conductivity approximately valid for noble metals like Au? What
      does this tell you about whether a good metal like gold or copper will behave the same to
      visible light as it will to microwaves? 

        % \textcolor{hwColor}{
        %   \\
        % }

      \item Using your graph and taking the high-frequency limit, describe the high-frequency behaviour
      of the metal. How do the group and phase velocities of the wave behave? Over what frequency
      range is the dispersion anomalous, and why?

        % \textcolor{hwColor}{
        %   \\
        % }

    \end{enumerate}
    

    \item \textbf{2 (20 points)} Suppose a material had absorption only in the frequency range $\omega < \left(\omega_0+\gamma\right)$
    and $\omega > \left(\omega_0-\gamma\right)$ with a constant imaginary part of the susceptibility $\chi^{''}(\omega)=\dfrac{\omega_p^2}{\omega_0^2}$ 
    in this regime. Use the Kramers-Kronig relations to derive the complex dielectric constant $\epsilon ~ \omega$. Graph the real and imaginary parts of the
    solution as a function of across the resonance frequency $\omega_0$. Qualitatively, is this result significantly
    different from the harmonic oscillator model far from the resonance?

        % \textcolor{hwColor}{
        %   \\
        % }

  \end{enumerate}

\end{document}
